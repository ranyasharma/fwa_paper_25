\begin{abstract}
    Historically, single-connection TCP wired networks demonstrate an inverse relationship between latency and throughput; however, our analysis reveals that this relationship does not consistently apply to fixed wireless contexts. Fixed wireless networks exhibit significantly higher variance in performance compared to wired networks, challenging conventional assumptions.
    Through extensive data collection from home-deployed Raspberry Pi devices across the United States, we analyze throughput and latency metrics from various ISPs, including Starlink, T-Mobile, and Verizon. Our findings reveal a lack of consistent correlation between throughput and latency, even within the same network type. In fixed wireless scenarios, throughput remains relatively stable, with occasional spikes that do not correspond to improved network performance, while latency fluctuates unpredictably. This demonstrates that throughput alone is not a reliable metric for assessing network quality in fixed wireless environments. Policymakers must reconsider its relevance when evaluating network performance and investing in ISPs. 
    \end{abstract}