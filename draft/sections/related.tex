\section{Related Work} \label{sec:related_work}
Numerous studies have explored the relationship between latency and throughput in network performance, 
particularly in the context of wired networks. Padhye et al. demonstrated an inverse
relationship  between throughput and latency for TCP connections, where congestion 
control mechanisms ensure that higher latencies result in reduced throughput. This 
model has been widely applied. 

Recent work has been focused on satellite networks. Michel et al. conducted an 
analysis of Starlink’s satellite internet performance, highlighting that Starlink can 
achieve high throughput and low latency under ideal conditions, but latency under load
is more variable. Their measurements showed that while Starlink's latency remains low
for close destinations (approximately 20 ms), it can spike to several hundred milliseconds under load. 

Kassem et al. further explored Starlink's performance using a browser-side measurement approach, 
collecting data from Starlink and non-Starlink users across multiple regions. 
Their findings revealed that while Starlink can provide high throughput, it also 
suffers from significant packet loss, latency variability, and sensitivity to weather. 
Notably, they observed that Starlink's bent-pipe architecture, where data travels from 
the user device to the satellite and then back to earth, is a dominant source of latency.

